% !TEX root = root.tex

% Sie sollten nur die Abkürzungen auflisten, die mit \ac definiert und auch benutzt wurden.
% Wenn gar keine Abkürzungen benutzt werden, dann muss der gesamte untere Code auskommentiert
% werden, ansonsten führt das zu einem Fehler.
%
% \acro{VERSIS}{Versicherungsinformationssystem\acroextra{ (Bestandsführungssystem)}}
% Ergibt in der Liste: VERSIS Versicherungsinformationssystem (Bestandsführungssystem)
% Im Text aber: \ac{VERSIS} -> Versicherungsinformationssystem (VERSIS)

% Hinweis: allgemein bekannte Abkürzungen wie z.B. bzw. u.a. müssen nicht ins Abkürzungsverzeichnis aufgenommen werden
% Hinweis: allgemein bekannte IT-Begriffe wie Datenbank oder Programmiersprache müssen nicht erläutert werden,
%          aber ggfs. Fachbegriffe aus der Domäne des Prüflings (z.B. Versicherung)

% Die Option (in den eckigen Klammern) enthält das längste Label oder
% einen Platzhalter der die Breite der linken Spalte bestimmt.
% \begin{acronym}[WWWWWW]
% 	\acro{VERSIS}{Versicherungsinformationssystem\acroextra{ (Bestandsführungssystem)}}
% \end{acronym}

\begin{acronym}[WWWWWW]
  \acro{MoSCoW}{Must Should Could Won't}
\end{acronym}
