% !TEX root = root.tex

\newcommand{\institutLogo}{logo-institut.pdf}
\newcommand{\institutLogoSmall}{logo-institut-small.pdf}
\newcommand{\institutName}{Hochschule Karlsruhe -- Technik und Wirtschaft}
\newcommand{\fakultaet}{Fakultät für Informatik und Wirtschaftsinformatik}
\newcommand{\studiengang}{Bachelor-Studiengang Informatik}
\newcommand{\studiengangKuerzel}{INFB}

\newcommand{\titel}{Bachelor-Thesis} % Die meisten PDF Viewer zeigen dies als den Namen des Dokuments an.
\newcommand{\untertitel}{Thema der Thesis}
\newcommand{\untertitelEnglisch}{Topic of Thesis}

\newcommand{\referentName}{Prof. Dr. Max Mustermann}
\newcommand{\referentEmail}{max.mustermann@institut.de}

\newcommand{\autorName}{Robin Hartmann}
\newcommand{\autorEmail}{contact.robin.hartmann@gmail.com}
\newcommand{\autorMatnr}{4711}

\newcommand{\zeitstempel}{\today}

\newcommand{\version}{1.0.0} % Hier haben Sie die Möglichkeit die Version dieses Dokuments anzugeben. Sie wird in den Dokumenteneigenschaften unter "Stichwörter" aufgeführt.
