% !TEX root = ../../root.tex

% Seitenränder -----------------------------------------------------------------
\setlength{\topskip}{\ht\strutbox} % behebt Warnung von geometry
\geometry{a4paper,left=20mm,right=20mm,top=25mm,bottom=37mm}

\usepackage[
  automark, % Kapitelangaben in Kopfzeile automatisch erstellen
  headsepline, % Trennlinie unter Kopfzeile
  ilines % Trennlinie linksbündig ausrichten
]{scrpage2}

% Kopf- und Fußzeilen ----------------------------------------------------------
\pagestyle{scrheadings}
% chapterpagestyle gibt es nicht in scrartcl
%\renewcommand{\chapterpagestyle}{scrheadings}
\clearscrheadfoot

% Kopfzeile
\renewcommand{\headfont}{\normalfont\sffamily} % Style der Kopf- und Fußzeile
\renewcommand{\pnumfont}{\normalfont\sffamily} % Style der Seitennummer
\ihead{\mbox{}\vfill\textit{\MakeUppercase{\headmark}}}
% \chead{}
\ohead{\includegraphics[scale=0.075]{\institutLogoSmall}}
\setlength{\headheight}{15mm} % Höhe der Kopfzeile
\setlength{\headsep}{9mm} % Abstand von Kopfzeile zu Text
\setlength{\skip\footins}{3mm} % Abstand von Fußnoten zu Text
%\setheadwidth[0pt]{textwithmarginpar} % Kopfzeile über den Text hinaus verbreitern (falls Logo den Text überdeckt)

% Fußzeile
% \ifoot{\autorName}
\cfoot{\pagemark}
% \ofoot{\autorName}
% \ofoot{\today}

% Überschriften nach DIN 5008 in einer Fluchtlinie
% ------------------------------------------------------------------------------

% Abstand zwischen Nummerierung und Überschrift definieren
\newcommand{\sectionHeadingSpace}{0.7cm}
\newcommand{\subsectionHeadingSpace}{1.05cm}
\newcommand{\subsubsectionHeadingSpace}{1.4cm}

% Einrückung der Einträge definieren
\newcommand{\sectionHeadingIndent}{0cm}
\newcommand{\subsectionHeadingIndent}{\sectionHeadingIndent + \sectionHeadingSpace}
\newcommand{\subsubsectionHeadingIndent}{\subsectionHeadingIndent + \subsectionHeadingSpace}

% Abschnittsüberschriften im selben Stil wie beim Inhaltsverzeichnis einrücken
\renewcommand*{\othersectionlevelsformat}[3]{
  \makebox[\sectionHeadingSpace][l]{#3\autodot}
}

% Für die Einrückung wird das Paket tocloft benötigt
%\cftsetindents{chapter}{\headingIndent}{\sectionHeadingSpace}
\cftsetindents{section}{\sectionHeadingIndent}{\sectionHeadingSpace}
\cftsetindents{subsection}{\subsectionHeadingIndent}{\subsectionHeadingSpace}
\cftsetindents{subsubsection}{\subsubsectionHeadingIndent}{\subsubsectionHeadingSpace}
\cftsetindents{figure}{\sectionHeadingIndent}{\sectionHeadingSpace}
\cftsetindents{table}{\sectionHeadingIndent}{\sectionHeadingSpace}

% Rückt Verzeichnis der Listings ein
\makeatletter
\renewcommand*{\l@lstlisting}{\@dottedtocline{1}{\sectionHeadingIndent}{\sectionHeadingSpace}}
\makeatother


% Allgemeines
% ------------------------------------------------------------------------------

\onehalfspacing % Zeilenabstand 1,5 Zeilen
\frenchspacing % erzeugt ein wenig mehr Platz hinter einem Punkt

% Quellcode-Ausgabe formatieren
\lstset{numbers=left, numberstyle=\tiny, numbersep=5pt, breaklines=true}
\lstset{emph={square}, emphstyle=\color{red}, emph={[2]root,base}, emphstyle={[2]\color{blue}}}

\counterwithout{footnote}{section} % Fußnoten fortlaufend durchnummerieren
\setcounter{tocdepth}{\subsubsectionlevel} % im Inhaltsverzeichnis werden die Kapitel bis zum Level der subsubsection übernommen
\setcounter{secnumdepth}{\subsubsectionlevel} % Kapitel bis zum Level der subsubsection werden nummeriert

% Aufzählungen anpassen
\renewcommand{\labelenumi}{\arabic{enumi}.}
\renewcommand{\labelenumii}{\arabic{enumi}.\arabic{enumii}.}
\renewcommand{\labelenumiii}{\arabic{enumi}.\arabic{enumii}.\arabic{enumiii}}

% Streckungfaktor von Tabellen
\renewcommand{\arraystretch}{1.3}
