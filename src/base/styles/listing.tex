% !TEX root = ../../root.tex

\usepackage{xcolor}

\definecolor{codegreen}{rgb}{0,0.6,0}
\definecolor{codegray}{rgb}{0.5,0.5,0.5}
\definecolor{codepurple}{rgb}{0.58,0,0.82}
\definecolor{backcolour}{rgb}{0.95,0.95,0.92}

\lstdefinestyle{custom}{
  backgroundcolor=\color{backcolour},
  commentstyle=\color{codegreen},
  keywordstyle=\color{magenta},
  ndkeywordstyle=\color{blue},
  numberstyle=\tiny\color{codegray},
  stringstyle=\color{codepurple},
  basicstyle=\ttfamily\footnotesize,
  breakatwhitespace=false,
  breaklines=true,
  captionpos=b,
  extendedchars=true,
  keepspaces=true,
  numbers=left,
  numbersep=9pt,
  numberstyle=\footnotesize,
  showspaces=false,
  showstringspaces=false,
  showtabs=false,
  tabsize=2
}

\lstdefinelanguage{typescript}{
  keywords={await, break, case, catch, do, else, export, if, import, return, switch, throw, while},
  ndkeywords={async, class, const, enum, false, function, get, implements, in, let, new, null, private, public, readonly, set, static, this, true, typeof, undefined, var, void},
  sensitive=false,
  comment=[l]{//},
  morecomment=[s]{/*}{*/},
  morestring=[b]',
  morestring=[b]"
}

\lstset{
  language=[Sharp]C, % Diese Sprache wird verwendet, falls beim Anlegen eines Listings keine angegeben wird
  % Eine Liste der unterstützten Sprachen lässt sich hier finden:
  % https://www.overleaf.com/learn/latex/code_listing#Supported_languages
  % Sprachdialekte werden angegeben mit [<Dialekt>]<Sprache>, bspw. [Sharp]C
  style=custom
}
