% !TEX root = ../root.tex
% Hinweis: der '%'-Kommentar am Ende von Befehlszeilen sorgt dafür, dass an der aufrufenden Stelle keine Leerstelle je Befehlszeile hier eingefügt wird

% Abkürzungen, ggfs. mit korrektem Leerraum
\newcommand{\bs}{$\backslash$\xspace}
\newcommand{\bzgl}{bzgl.\xspace}
\newcommand{\bspw}{bspw.\xspace}
\newcommand{\bzw}{bzw.\xspace}
\newcommand{\ca}{ca.\xspace}
\newcommand{\dahe}{\mbox{d.\,h.}\xspace}
\newcommand{\etc}{etc.\xspace}
\newcommand{\eur}[1]{\mbox{\num{#1}\,\texteuro}\xspace}
\newcommand{\evtl}{evtl.\xspace}
\newcommand{\ggfs}{ggfs.\xspace}
\newcommand{\Ggfs}{Ggfs.\xspace}
\newcommand{\gqq}[1]{\glqq{}#1\grqq{}}
\newcommand{\inkl}{inkl.\xspace}
\newcommand{\insb}{insb.\xspace}
\newcommand{\ua}{\mbox{u.\,a.}\xspace}
\newcommand{\usw}{usw.\xspace}
\newcommand{\Vgl}{Vgl.\xspace}
\newcommand{\zB}{\mbox{z.\,B.}\xspace}

% Befehle für häufig anfallende Aufgaben
\newcommand{\includegraphicsKeepAspectRatio}[2]{\includegraphics[width=#2\textwidth,height=#2\textheight,keepaspectratio]{#1}} % Parameter: file, scale
\newcommand{\Zitat}[2][\empty]{\ifthenelse{\equal{#1}{\empty}}{\citep{#2}}{\citep[#1]{#2}}}
\newcommand{\Autor}[1]{\textsc{#1}} % zum Ausgeben von Autoren
\newcommand{\itemd}[2]{\item{\textbf{#1}}\\{#2}} % erzeugt ein Listenelement mit fetter Überschrift

\newcommand{\secref}[1]{\autoref{sec:#1}: \nameref{sec:#1}} % erzeugt eine Referenz zu einer section im Format "Abschnitt/Unterabschnitt <nummer>: <name>"
\newcommand{\appref}[1]{\appendixname{}~\ref{app:#1}: \nameref{app:#1} \vpageref{app:#1}} % erzeugt eine Referenz zu einem Anhang im Format "Anhang <nummer>: <name> auf Seite <seite>"
\newcommand{\figref}[1]{\autoref{fig:#1}} % erzeugt eine Referenz zu einer Abbildung im Format "Abbildung <nummer>"
\newcommand{\tabref}[1]{\autoref{tab:#1}} % erzeugt eine Referenz zu einer Tabelle im Format "Tabelle <nummer>"
\newcommand{\lstref}[1]{\autoref{lst:#1}} % erzeugt eine Referenz zu einem Listing im Format "Listing <nummer>"
\newcommand{\itemref}[1]{\autoref{enum:#1}} % erzeugt eine Referenz zu Stichpunkt im Format "Punkt <nummer>"

% Hilfsbefehl für label (Parameter: [label], section, prefix)
\newcommand{\createlabel}[3][]{%
  \ifthenelse{\isempty{#1}}%
  {\label{#3:#2}}%
  {\label{#3:#1}}%
}

% Hilfsbefehle für label (Parameter: [label], sectionname, prefix, sectiontype)
\newcommand{\labeledgeneric}[4][]{#4{#2}\createlabel[#1]{#2}{#3}}
\newcommand{\labeledgenericrootsection}[3][]{\labeledgeneric[#1]{#2}{#3}{\section}}
\newcommand{\labeledgenericsubsection}[3][]{\labeledgeneric[#1]{#2}{#3}{\subsection}}
\newcommand{\labeledgenericsubsubsection}[3][]{\labeledgeneric[#1]{#2}{#3}{\subsubsection}}

% Befehle für normale section mit label (Parameter: [label], section)
\newcommand{\labeledsection}[2][]{\labeledgenericrootsection[#1]{#2}{sec}}
\newcommand{\labeledsubsection}[2][]{\labeledgenericsubsection[#1]{#2}{sec}}
\newcommand{\labeledsubsubsection}[2][]{\labeledgenericsubsubsection[#1]{#2}{sec}}

% Befehle für appendix section mit label (Parameter: [label], section)
\newcommand{\labeledappendix}[2][]{\labeledgenericsubsection[#1]{#2}{app}}
\newcommand{\labeledsubappendix}[2][]{\labeledgenericsubsubsection[#1]{#2}{app}}

% Befehle für item aus enum mit label (Parameter: label)
\newcommand{\labeleditem}[1]{\item\label{enum:#1}}

\newcommand{\footcite}[1]{\footnote{\Vgl \cite{#1}}} % Erstellt eine Fußnote mit Zitierung in der Form "Vgl. <zitierung>"

% fügt Tabellen aus einer TEX-Datei ein
\newcommand{\ctablex}[3]{ % Parameter: caption, label, file
  \begin{table}[htbp]
    \centering
    \singlespacing
    \input{tables/#3}
    \caption{#1}
    \label{tab:#2}
  \end{table}
}

% Alternative Funktion zum Einfügen von Tabellen, die label=caption setzt
\newcommand{\ctable}[2]{ % Parameter: caption, file
  \ctablex{#1}{#1}{#2}
}

% Alternative Funktion zum Einfügen von Tabellen in den Appendix, die label=caption setzt und die die Caption auch als Überschrift hinzufügt
\newcommand{\ctableapp}[2]{ % Parameter: caption, file
  \labeledappendix{#1}
  \ctable{#1}{#2}
}

% Alternative Funktion zum Einfügen von Tabellen, die caption und label weglässt
\newcommand{\ctablen}[1]{ % Parameter: file
  \begin{center}
    \singlespacing
    \input{tables/#1}
  \end{center}
}

% fügt Bilder ein, sodass sie über \figref referenziert werden können
\newcommand{\cfigurex}[5]{ % Parameter: caption, label, file, width, trim
  % trim ist wird angegeben mit {<links>px <unten>px <rechts>px <oben>px}
  % bspw. {200px 80px 200px 50px}
  \begin{figure}[htb]
    \centering
    \includegraphics[keepaspectratio, clip, width=#4, trim={#5}]{#3}
    \caption{#1}
    \label{fig:#2}
  \end{figure}
}

% Alternative Funktion zum Einfügen von Bildern, die label=caption setzt
\newcommand{\cfigure}[4]{ % Parameter: caption, file, width, trim
  \cfigurex{#1}{#1}{#2}{#3}{#4}
}

% Alternative Funktion zum Einfügen von Bildern in den Appendix, die label=caption setzt und die die Caption auch als Überschrift hinzufügt
\newcommand{\cfigureapp}[4]{ % Parameter: caption, file, width, trim
  \labeledappendix{#1}
  \cfigurex{#1}{#1}{#2}{#3}{#4}
}

% Alternative Funktion zum Einfügen von Bildern, die caption und label weglässt
\newcommand{\cfiguren}[3]{ % Parameter: file, width, trim
  \begin{figure}[htb]
    \centering
    \includegraphics[keepaspectratio, clip, width=#2, trim={#3}]{#1}
  \end{figure}
}

% fügt Quellcode in den Anhang ein (Parameter: [language], path)
\newcommand{\quellcode}[2][]{%
  \labeledsubappendix{Listing der Datei #2}%
  \ifthenelse{\isempty{#1}}%
  {\lstinputlisting[caption=#2]{Listings/#2}}%
  {\lstinputlisting[caption=#2, language=#1]{Listings/#2}}%
}

% einfaches Wechseln der Schrift, z.B.: \changefont{cmss}{sbc}{n}
\newcommand{\changefont}[3]{\fontfamily{#1} \fontseries{#2} \fontshape{#3} \selectfont}

% Verwendung analog zu \includegraphics
% Parameter: file, padding-x, padding-y
% Beispiel: \includegraphicstotab[scale=0.5]{example.png}{2px}{2px}
\newlength{\myx} % Variable zum Speichern der Bildbreite
\newlength{\myy} % Variable zum Speichern der Bildhöhe
\newcommand\includegraphicstotab[4][\relax]{%
  \settowidth{\myx}{\includegraphics[{#1}]{#2}}%
  \settoheight{\myy}{\includegraphics[{#1}]{#2}}%
  \parbox[c][\myy + #4][c]{\myx + #3}{%
    \includegraphics[{#1}]{#2}}%
}


% verschiedene Befehle um Wörter semantisch auszuzeichnen ----------------------
\newcommand{\Index}[2][\empty]{\ifthenelse{\equal{#1}{\empty}}{\index{#2}#2}{\index{#1}#2}}
\newcommand{\Fachbegriff}[2][\empty]{\ifthenelse{\equal{#1}{\empty}}{\textit{\Index{#2}}}{\textit{\Index[#1]{#2}}}}
\newcommand{\NeuerBegriff}[2][\empty]{\ifthenelse{\equal{#1}{\empty}}{\textbf{\Index{#2}}}{\textbf{\Index[#1]{#2}}}}

\newcommand{\Ausgabe}[1]{\texttt{#1}}
\newcommand{\Eingabe}[1]{\texttt{#1}}
\newcommand{\Code}[1]{\texttt{#1}}
\newcommand{\Datei}[1]{\texttt{#1}}

\newcommand{\Assembly}[1]{\textsf{#1}}
\newcommand{\Klasse}[1]{\textsf{#1}}
\newcommand{\Methode}[1]{\textsf{#1}}
\newcommand{\Attribut}[1]{\textsf{#1}}

\newcommand{\Datentyp}[1]{\textsf{#1}}
\newcommand{\XMLElement}[1]{\textsf{#1}}
\newcommand{\Webservice}[1]{\textsf{#1}}

\newcommand{\Refactoring}[1]{\Fachbegriff{#1}}
\newcommand{\CodeSmell}[1]{\Fachbegriff{#1}}
\newcommand{\Metrik}[1]{\Fachbegriff{#1}}
\newcommand{\DesignPattern}[1]{\Fachbegriff{#1}}
